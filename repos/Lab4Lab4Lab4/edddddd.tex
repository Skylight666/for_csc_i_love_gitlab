\documentclass[10pt]{article}
\usepackage[T2A]{fontenc}
\usepackage[utf8]{inputenc}
\usepackage[russian]{babel}
\usepackage{graphicx}
\usepackage{mathtools}
\usepackage{mathtext}
\usepackage{graphicx}
\usepackage{wrapfig}
\usepackage{caption}
\usepackage{hyphenat}
\hyphenation{ма-те-ма-ти-ка вос-ста-нав-ли-вать}
\usepackage[left=45mm, top=30mm, right=63mm, bottom=0mm, nofoot]{geometry}
\begin{document}
		\footnotesize	

		\noindent
		76 ЛАГРАНЖЕВА МЕХАНИКА НА МНОГООБРАЗИЯХ ГЛ. 4 

		\noindent
		такой же, как в одномерной системе с кинетической энергией


		\footnotesize	

		\begin{equation*}
			T_0 = \frac{M}{2} q^2 , \hspace{1mm} M=mr^2
		\end{equation*}
		
		\noindent
		и с потенциальной энергией

		\begin{equation*} 
			V = A \cos{q} - B \sin^2{q}, 
			\hspace{1mm} A=mgr, 
			\hspace{1mm} B=\frac{m}{2} \omega^2 r^2
		\end{equation*}
		
		\noindent
		Вид фазового портрета зависит \hspace{0.6mm} от \hspace{0.6mm} соотношения между А и В. При 2В < А \\
		(т. е. при таком медленном варщении окружности, что $\omega^2$r < g) нижнее поло-\\
		жение бусинки (q=$\pi$) устойчиво и характер движения в общем такой же, как
		в случае математического маятника ($\omega$=0). 
		\par
		При 2B > A, т.е.  при достаточно быстром вращении окружности, нижнее
		положение бусинки становится неустойчивым, зато появляются два устойчивых
		
		\begin{figure}
			\includegraphics{screenshot001.png}
		\end{figure}

\end{document}